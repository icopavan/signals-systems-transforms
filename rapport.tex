%
%  untitled
%
%  Created by Jesper Josefsson on 2011-10-10.
%  Copyright (c) 2011 __MyCompanyName__. All rights reserved.
%
\documentclass[]{article}

% Use utf-8 encoding for foreign characters
\usepackage[utf8]{inputenc}

% Setup for fullpage use
\usepackage{fullpage}

% Uncomment some of the following if you use the features
%
% Running Headers and footers
\usepackage{fancyhdr}

% Multipart figures
%\usepackage{subfigure}

% More symbols
\usepackage{amsmath}
\usepackage{amssymb}
\usepackage{latexsym}

% Surround parts of graphics with box
\usepackage{boxedminipage}

% Package for including code in the document
\usepackage{listings}

% If you want to generate a toc for each chapter (use with book)
\usepackage{minitoc}

% This is now the recommended way for checking for PDFLaTeX:
\usepackage{ifpdf}

%\newif\ifpdf
%\ifx\pdfoutput\undefined
%\pdffalse % we are not running PDFLaTeX
%\else
%\pdfoutput=1 % we are running PDFLaTeX
%\pdftrue
%\fi

\ifpdf
\usepackage[pdftex]{graphicx}
\else
\usepackage{graphicx}
\fi
\title{SSY080 \\ Inlämningsuppgift}
\author{
Linus Oleander - 880613 - 4873 \\
Jesper Josefsson - 860409 - 5276 
}

\date{2011-10-10}

\begin{document}

\ifpdf
\DeclareGraphicsExtensions{.pdf, .jpg, .tif}
\else
\DeclareGraphicsExtensions{.eps, .jpg}
\fi

\maketitle

\section{Bakgrund}
Uppgiften går ut på att genomföra ett antal experiment gällande generering och behandling av signaler med Matlab.

\section{Generering av fyrkantsvåg med hjälp av Fourierserie - (3.1)}
Den första uppgiften var att ta fram ett slutet uttryck för Fourierseriekoefficienterna $A_k$ och $B_k$.\\
Vi använde följande samband:\\
\begin{align*}
	C_k &= \frac{1}{T} \int_0^T \! x(t) e^{-jk\omega_0 t}\, \mathrm{d} t =\\
	    &= \frac{1}{T} \left(
	          \int_0^{\frac{T}{2}} \! e^{-jk\omega_0 t}\, \mathrm{d} t
	          - \int_{\frac{T}{2}}^T \! e^{-jk\omega_0 t}\, \mathrm{d} t \right) =\\
      &= \frac{1}{-jk\omega_0 T} \left(
            \left[e^{-jk\omega_0 t}\right]_0^{\frac{T}{2}}
            - \left[e^{-jk\omega_0 t}\right]_{\frac{T}{2}}^T
          \right) = \\
      &= \left[
          \omega = \frac{2\pi}{T} \Rightarrow \frac{\omega_0 T}{2} = \pi ,\;  \omega_0 T = 2\pi
        \right] = \\
      &= \frac{1}{-jk\omega_0 T} \left(
          2e^{-jk\pi} - e^{-jk2\pi} - 1
        \right) = \\
      &=  \left[
          \begin{array}{ll}
            2e^{-jk\pi} &=
              \left\{
              	\begin{array}{ll}
              		-2  & ,k\text{ udda} \\
              		2 & ,k\text{ jämn} 
              	\end{array}
              \right. \\
            e^{-jk2\pi} &= 1
          \end{array}
          \right] = \\ \\
  C_k &= \left\{ \begin{array}{ll}
            \frac{-4}{-jk2\pi} = \frac{2}{jk\pi} = -\frac{2j}{k\pi}&,k\text{ udda} \\
            0 &,k\text{ jämn}
          \end{array} \right.\\
\end{align*}


\bibliographystyle{plain}
\bibliography{}
\end{document}
